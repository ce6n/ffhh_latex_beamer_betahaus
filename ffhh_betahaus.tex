\documentclass[aspectratio=43]{beamer}
\usetheme{intridea}  %% Themenwahl

\usepackage[ngerman]{babel} 
\usepackage[T1]{fontenc}    % richtige Silbentrennung
\usepackage[utf8]{inputenc} % Umlaute etc.!
\usepackage{eurosym}
\title{Freifunk Hamburg}
\author{Freifunk Hamburg}
\date{\today}

\begin{document}
\maketitle
\frame{\tableofcontents[currentsection]}

%2
\section{Abschnitt 1}
\begin{frame} %%Eine Folie
  \frametitle{Was ist freifunk.net} %%Folientitel
  hamburg.freifunk ist eine nichtkommerzielle Initiative, die in Zusammenarbeit mit anderen Organisationen und Gruppen die Idee freier Netzwerke fördert
  \begin{itemize}
	\item Digitale Gesellschaft	
  \end{itemize}
\end{frame}

%2
\begin{frame}
\begin{itemize}
	\item \textbf{frei} wird dabei verstanden als:
	\begin{itemize}
		\item öffentlich - jede\_m zugänglich
		\item nicht kommerziell
		\item im Besitz der Gemeinschaft
		\item netzneutral - keine Manipulation der Datenströme
	\end{itemize}
  \end{itemize}
\end{frame}

%3
\begin{frame}{Ziel des Projekts}
	\begin{itemize}
		\item Verbreitung offener WLAN-Netzwerke
		\item Zugangshürden zum Internet minimieren
		\item Aufklärung und Sensibilisierung zum Thema ``Kommunikations- und Informationsfreiheit''
		\item Menschen dazu befähigen, eigene Netze aufzubauen und zu betreiben
		\item Soziale Strukturen bilden und unterstützen %?
	\end{itemize}
\end{frame}

%5
\begin{frame}{Warum WLAN?}
	\begin{itemize}
		\item Mit WLAN können Daten mobil mit hoher Bandbreite gesendet und empfangen werden
		\item Die Kosten für WLAN-Hardware sind gering und es entstehen kaum Betriebskosten (Router ab \EUR{15}, Strom \EUR{10} im Jahr)
		\item WLAN kann auch dort eingesetzt werden, wo es keine Kabel gibt oder eine Kabelverbindung zu teuer ist. %[Parks, Entwicklungsländer, etc...]
		\item (WLAN kann von jedem lizenzfrei eingesetzt werden) %weg?
	\end{itemize}
\end{frame}

%10
\begin{frame}{Das Konzept von ``Mesh Netzwerken''} % oder Networks?
	Sich selbst organisierende Netzwerke
\end{frame}

%11
\begin{frame}{Das Konzept von ``Mesh Netzwerken''} % same here
	(to mesh = Englisch: vermaschen)

	Jeder Accesspoint in einem Netzwerk wird automatisch zu einem aktiven Knoten fuer andere
	\begin{itemize}
		\item $A$ kann $B$ erreichen und $B$ erreicht $C$
		\item ueber ``ad-hoc routing''-Protokolle tauschen die Geraete selbststaendig Informationen darueber aus, wer wen erreichen kann
		\item $A$ erreicht automatisch $B$, wenn\\
		$A$ Kontakt zu $B$ und\\
		$B$ Kontakt zu $C$ hat
	\end{itemize}
\end{frame}

%12
\begin{frame}{Vorraussetzung ist freier Datentransit} % Titel aendern
	Damit Informationen von einem Ende des Netzes ($A$) zum anderen gelangen koennen ($C$), muss die dazwischen liegende Knotenpunkte ($B$) den ungehinderten Transfer von Daten erlauben!\\
	Das machen die Router im Freifunk.
\end{frame}

%18
\begin{frame}{Das Netz waechst}

\end{frame}

%19
\begin{frame}{Netzwerke verbinden sich untereinander}

\end{frame}

%23
\begin{frame}{Wachstum garantiert}
	\begin{itemize}
		\item Durch niedrige Einstiegshürde kann fast jede\_r Interessierte einfach teilnehmen (\EUR{20})
		\item In fast jeder deutschen Großstadt gibt es eine lokale freifunk-Community
		\item Das hamburger Freifunkprojekt besteht bereits aus XX Knoten und wächst stetig weiter %TODO abgewandelt, so ok? CCC erwaehnen?
		\item Auch in vielen ländlichen Regionen sind bereits Freifunknetze entstanden
		\item Viele Gastronomiebetriebe in Hamburg nutzen Freifunk, um ihren Gästen ein kostenloses und anonymisiertes WLAN anzubieten, ohne dafür Geld an kommerzielle WLAN Anbieter zahlen zu müssen %TODO stimmt nicht. rauslassen? abwandeln?
	\end{itemize}
\end{frame}

%27
\begin{frame}{Welche Regeln sollte man beim Mitmachen beachten?}
	\begin{itemize}
		\item Dienstbereitstellung erfolgt ausschließlich auf freiwilliger Basis
		\item Es gibt keinerlei Nutzungsgarantie
		\item Entgelte (z.B. für Wartung/Pflege der WLAN-Router, DSL) nachbarschaftlich regeln, nie für Internetangebot kommerziell Geld verlangen
		\item Haftung - generell gilt: verantwortlich ist immer, wer die unerlaubte Handlung vollführt, nicht wer den hamburger Freifunkknoten bereitstellt
	\end{itemize}
	\it{Freifunkknoten = Freifunkrouter} %TODO raus?
\end{frame}

%6
\begin{frame}{Vision}
	\begin{itemize}
		\item Mit WLAN vernetzen sich nicht nur Menschen in einem Haus sondern ganze Stadtteile, Dörfer und Regionen %nur Stadtteile?
		\item Ein eigenes, unabhängiges Netzwerk aufbauen, um damit zum Beispiel:
		\begin{itemize}
			\item Kostenlos untereinander Daten auszutauschen und zu telefonieren (Voice-over-IP)
			\item Lizenzfreies Community Radio betreiben und hören
			\item Bildtelefonier und Videobroadcasting (Web-TV) nutzen %hier vielleicht auch lieber dann telefonieren auffuehren
			\item <deine Idee hier> %weg?
		\end{itemize}
	\end{itemize}
\end{frame}

\end{document}