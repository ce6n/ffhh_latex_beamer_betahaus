\documentclass[aspectratio=43]{beamer}
\usetheme{intridea}  %% Themenwahl

\usepackage[ngerman]{babel} 
\usepackage[T1]{fontenc}    % richtige Silbentrennung
\usepackage[utf8]{inputenc} % Umlaute etc.!
\usepackage{eurosym}
\usepackage{tikz}

\usetikzlibrary{arrows,decorations.pathmorphing,backgrounds,fit,positioning,shapes.symbols,chains}

\title{Freifunk Hamburg}
\author{Freifunk Hamburg}
\date{\today}

\begin{document}
\maketitle
\frame{\tableofcontents[currentsection]}

%2
\section{Abschnitt 1}
\begin{frame}{Was ist freifunk.net?}
  hamburg.freifunk ist eine nichtkommerzielle Initiative, die in Zusammenarbeit mit anderen Organisationen und Gruppen die Idee freier Netzwerke fördert
  \begin{itemize}
	\item Digitale Gesellschaft
	\item CCC
	\item MOAR %TODO MOAR
  \end{itemize}
\end{frame}

%2
\begin{frame}
\begin{itemize}
	\item \textbf{frei} wird dabei verstanden als:
	\begin{itemize}
		\item Öffentlich - jedem zugänglich
		\item Nicht kommerziell
		\item Im Besitz der Gemeinschaft
		\item Netzneutral - keine Manipulation der Datenströme
	\end{itemize}
	\item Mit \textbf{Vernetzung} meinen wir:
	\begin{itemize}
		\item Kommunikation zwischen Menschen unter Verwendung digitaler Medien (Computer, Handys, Datennetze)
	\end{itemize}
  \end{itemize}
\end{frame}

%3
\begin{frame}{Ziel des Projekts}
	\begin{itemize}
		\item Verbreitung offener WLAN-Netzwerke
		\item Zugangshürden zum Internet minimieren
		\item Aufklärung und Sensibilisierung zum Thema ``Kommunikations- und Informationsfreiheit''
		\item Menschen dazu befähigen, eigene Netze aufzubauen und zu betreiben
		\item Soziale Strukturen bilden und unterstützen %TODO ?
	\end{itemize}
\end{frame}

%4
\begin{frame}{Freie Netzwerke - Wozu?} %TODO hatten wir eigentlich rausgelassen, aber ich finde min. die beiden ersten punkte interessant. vlt nimmt man die noch fuer eine andere folie?
	\begin{itemize}
		\item Die Informations- und Kommunikationsfreiheit im Internet wird zunehmend eingeschränkt
		\item Trotz des Slogans ``Internet für alle'' gibt es Anzeichen einer sich verfestigenden digitalen Kluft - ärmere, weniger technisch versierte und ältere Menschen nehmen wenig oder gar nicht am sogenannten Informationszeitalter teil
		\item In dünn besiedelten und strukturschwachen Gebieten (``areas of market failure'') werden keine (bezahlbaren) Breitbandanschlüsse angeboten
		\item Echte ``e-democracy'' soll von ``unten'' kommen
	\end{itemize}
\end{frame}

%5
\begin{frame}{Warum WLAN?}
	\begin{itemize}
		\item Mit WLAN können Daten mobil mit hoher Bandbreite gesendet und empfangen werden
		\item Die Kosten für WLAN-Hardware sind gering und es entstehen kaum Betriebskosten (Router ab \EUR{15}, Strom \EUR{10} im Jahr)
		\item WLAN kann auch dort eingesetzt werden, wo es keine Kabel gibt oder eine Kabelverbindung zu teuer ist. %[Parks, Entwicklungsländer, etc...]
		\item (WLAN kann von jedem lizenzfrei eingesetzt werden) %weg?
	\end{itemize}
\end{frame}

%10
\begin{frame}{Das Konzept von ``Mesh Netzwerken''} % oder Networks?
	Sich selbst organisierende Netzwerke
\end{frame}

%11
\begin{frame}{Das Konzept von ``Mesh Netzwerken''} % same here
	(to mesh = Englisch: vermaschen)

	Jeder Accesspoint in einem Netzwerk wird automatisch zu einem aktiven Knoten für andere
	\begin{columns}[c]
		\begin{column}[c]{0.4\textwidth}
			\begin{tikzpicture}
				\definecolor{outerCircleColour}{RGB}{220, 0, 103}
				\definecolor{innerCircleColour}{RGB}{255, 203, 18}
				\tikzstyle{vertex}=[circle, draw, color=outerCircleColour, ultra thick, minimum size=2.1cm]
				\tikzstyle{place}=[circle, draw, color=innerCircleColour, fill=innerCircleColour, minimum size=1.5mm, inner sep=0pt]

				\node [vertex] (a) {};
				\node [vertex, xshift=1.7cm, yshift=0.6cm] (b) {};
				\node [vertex, xshift=2.8cm, yshift=-0.5cm] (c) {};

				\node [place, label=above:$A$] (A) {};
				\node [place, xshift=1.7cm, yshift=0.6cm, label=above:$B$] (B) {};
				\node [place, xshift=2.8cm, yshift=-0.5cm, label=right:$C$] (C) {};

				\path[-, thick, color=gray]
				(A) edge (B)
				(B) edge (C);
			\end{tikzpicture}
		\end{column}
		\begin{column}[c]{0.6\textwidth}
			\begin{itemize}
				\item $A$ kann $B$ erreichen und $B$ erreicht $C$
				\item Über ``ad-hoc routing''-Protokolle tauschen die Geräte selbstständig Informationen darüber aus, wer wen erreichen kann
				\item $A$ erreicht automatisch $B$, wenn\\
				$A$ Kontakt zu $B$ und\\
				$B$ Kontakt zu $C$ hat
			\end{itemize}
		\end{column}
	\end{columns}
\end{frame}

%12
\begin{frame}{Vorraussetzung ist freier Datentransit} %TODO Titel aendern
	\begin{columns}[c]
		\begin{column}[c]{.4\textwidth}
			\begin{tikzpicture}
				\definecolor{outerCircleColour}{RGB}{220, 0, 103}
				\definecolor{innerCircleColour}{RGB}{255, 203, 18}
				\tikzstyle{vertex}=[circle, draw, color=outerCircleColour, ultra thick, minimum size=2.1cm]
				\tikzstyle{place}=[circle, draw, color=innerCircleColour, fill=innerCircleColour, minimum size=1.5mm, inner sep=0pt]

				\node [vertex] (a) {};
				\node [vertex, xshift=1.7cm, yshift=0.6cm] (b) {};
				\node [vertex, xshift=2.8cm, yshift=-0.5cm] (c) {};

				\node [place, label=above:$A$] (A) {};
				\node [place, xshift=1.7cm, yshift=0.6cm, label=above:$B$] (B) {};
				\node [place, xshift=2.8cm, yshift=-0.5cm, label=right:$C$] (C) {};

				\path[-, thick, color=gray]
				(A) edge (B)
				(B) edge (C)
				(A) edge[dashed] (C);
			\end{tikzpicture}
		\end{column}
		\begin{column}[c]{.6\textwidth}
			Damit Informationen von einem Ende des Netzes ($A$) zum anderen gelangen können ($C$), muss die dazwischen liegende Knotenpunkte ($B$) den ungehinderten Transfer von Daten erlauben!\\
			Das machen die Router im Freifunk.
		\end{column}
	\end{columns}
	\end{frame}

%18
\begin{frame}{Das Netz waechst}

\end{frame}

%19
\begin{frame}{Netzwerke verbinden sich untereinander}

\end{frame}

%13
\begin{frame}{Praktische Umsetzung}
	\begin{itemize}
		\item Sehr häufig verwendet: TP-Link 741nd (ca. \EUR{15})
		\item Austausch der vorhandenen Software durch Linux Firmware, die man von unserer Webseite herunterladen kann
		\item Antennen abschraub- und austauschbar
		\item Auf Wunsch erhältst du von uns ein fertig konfiguriertes Gerät, dass nur noch zu Hause angeschlossen werden muss
		\item Vorgefertigte Software auch für weitere Router
	\end{itemize}
\end{frame}

%23
\begin{frame}{Wachstum garantiert}
	\begin{itemize}
		\item Durch niedrige Einstiegshürde kann fast jede\_r Interessierte einfach teilnehmen (\EUR{15})
		\item In fast jeder deutschen Großstadt gibt es eine lokale freifunk-Community
		\item Das hamburger Freifunkprojekt besteht bereits aus XX Knoten und wächst stetig weiter %TODO abgewandelt, so ok? CCC erwaehnen?
		\item Auch in vielen ländlichen Regionen sind bereits Freifunknetze entstanden
		\item Viele Gastronomiebetriebe in Hamburg nutzen Freifunk, um ihren Gästen ein kostenloses und anonymisiertes WLAN anzubieten, ohne dafür Geld an kommerzielle WLAN Anbieter zahlen zu müssen %TODO stimmt nicht. rauslassen? abwandeln?
	\end{itemize}
\end{frame}

%20
\begin{frame}{Die Idee}
	Vernetzung von Kulturstätten und Aufbau eines stadtweiten Intranets
	\begin{itemize}
		\item Verschiedene Kultureinrichtungen werden zu WLAN-Knotenpunkten im lokalen Raum
		\item Die einzelnen Einrichtungen vernetzen sich untereinander
		\item Es entsteht ein eigenes stadtweites Intranet mit lokalen Zugängen ins Internet
		\item Kultureinrichtungen können ein eigenes Programm (Communityradio, Streaming-Media) im Intranet anbieten
		\item Jeder Mensch kann Teil dieses Netzes sein ohne monatliche Kosten
	\end{itemize}
\end{frame}

%27
\begin{frame}{Welche Regeln sollte man beim Mitmachen beachten?}
	\begin{itemize}
		\item Dienstbereitstellung erfolgt ausschließlich auf freiwilliger Basis
		\item Es gibt keinerlei Nutzungsgarantie
		\item Entgelte (z.B. für Wartung/Pflege der WLAN-Router, DSL) nachbarschaftlich regeln, nie für Internetangebot kommerziell Geld verlangen
		\item Haftung - generell gilt: verantwortlich ist immer, wer die unerlaubte Handlung vollführt, nicht wer den hamburger Freifunkknoten bereitstellt
	\end{itemize}
	\it{Freifunkknoten = Freifunkrouter} %TODO raus?
\end{frame}

\begin{frame}{Wie kann ich Freifunk Hamburg unterstuetzen?}
	\begin{itemize}
		\item ...
	\end{itemize}
\end{frame}

%6
\begin{frame}{Vision}
	\begin{itemize}
		\item Mit WLAN vernetzen sich nicht nur Menschen in einem Haus sondern ganze Stadtteile, Dörfer und Regionen %nur Stadtteile?
		\item Ein eigenes, unabhängiges Netzwerk aufbauen, um damit zum Beispiel:
		\begin{itemize}
			\item Kostenlos untereinander Daten auszutauschen und zu telefonieren (Voice-over-IP)
			\item Lizenzfreies Community Radio betreiben und hören
			\item Bildtelefonier und Videobroadcasting (Web-TV) nutzen %TODO hier vielleicht auch lieber dann telefonieren auffuehren
			\item <deine Idee hier> %TODO weg?
		\end{itemize}
	\end{itemize}
\end{frame}

\end{document}