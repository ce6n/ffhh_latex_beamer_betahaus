\documentclass[aspectratio=43]{beamer}
\usetheme{intridea}  %% Themenwahl

\usepackage[ngerman]{babel} 
\usepackage[T1]{fontenc} 				% richtige Silbentrennung
\usepackage[utf8]{inputenc}    % Umlaute etc.! 
\title{Freifunk Hamburg}
\author{Freifunk Hamburg}
\date{\today}

\begin{document}
\maketitle
\frame{\tableofcontents[currentsection]}

\section{Abschnitt 1}
\begin{frame} %%Eine Folie
  \frametitle{Was ist freifunk.net} %%Folientitel
  hamburg.freifunk ist eine nichtkommerzielle Initiative, die in Zusammenarbeit mit anderen Organisationen und Gruppen die Idee freier Netzwerke fördert
  \begin{itemize}
	\item Digitale Gesellschaft	
  \end{itemize}
\end{frame}

%2
\begin{frame}
\begin{itemize}
	\item \textbf{frei} wird dabei verstanden als:
	\begin{itemize}
		\item öffentlich - jede\_m zugänglich
		\item nicht kommerziell
		\item im Besitz der Gemeinschaft
		\item netzneutral - keine Manipulation der Datenströme
	\end{itemize}
  \end{itemize}
\end{frame}


%23
\begin{frame}{Wachstum garantiert}
	\begin{itemize}
		\item Durch niedrige Einstiegshuerde kann fast jede\_r Interessierte einfach teilnehmen (20 €)
		\item In fast jeder deutschen Groszstadt gibt es eine lokale freifunk-Community
		\item Das hamburger Freifunkprojekt besteht bereits aus XX Knoten und waechst stetig weiter %TODO abgewandelt, so ok? CCC erwaehnen?
		\item Auch in vielen laendlichen Regionen sind bereits Freifunknetze entstanden
		\item Viele Gastronomiebetriebe in Hamburg nutzen Freifunk, um ihren Gaesten ein kostenloses und anonymisiertes WLAN anzubieten, ohne dafuer Geld an kommerzielle WLAN Anbieter zahlen zu muessen %TODO stimmt nicht. rauslassen? abwandeln?
	\end{itemize}
\end{frame}

%27
\begin{frame}{Welche Regeln sollte man beim Mitmachen beachten?}
	\begin{itemize}
		\item Dienstbereitstellung erfolgt ausschließlich auf freiwilliger Basis
		\item Es gibt keinerlei Nutzungsgarantie
		\item Entgelte (z.B. für Wartung/Pflege der WLAN-Router, DSL) nachbarschaftlich regeln, nie für Internetangebot kommerziell Geld verlangen
		\item Haftung - generell gilt: verantwortlich ist immer, wer die unerlaubte Handlung vollführt, nicht wer den hamburger Freifunkknoten bereitstellt
	\end{itemize}
	\it{Freifunkknoten = Freifunkrouter} %TODO raus?
\end{frame}

\end{document}